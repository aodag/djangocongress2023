% Created 2023-09-24 日 23:21
% Intended LaTeX compiler: pdflatex
\documentclass[presentation]{beamer}
\usepackage{luatexja}

                   \usepackage{luatexja}
                   \usepackage{luatexja-preset}
\usetheme{Madrid}
\usecolortheme{beetle}
\author{Atsushi Odagiri}
\date{2023-10-07}
\title{django migrationで学ぶデータベース設計}
\hypersetup{
 pdfauthor={Atsushi Odagiri},
 pdftitle={django migrationで学ぶデータベース設計},
 pdfkeywords={},
 pdfsubject={},
 pdfcreator={Emacs 29.1 (Org mode 9.6.6)}, 
 pdflang={English}}
\begin{document}

\maketitle
\begin{frame}{Outline}
\tableofcontents
\end{frame}


\section{正規化}
\label{sec:org155ef6d}
\begin{frame}[label={sec:org7297f08}]{正規化しろ}
\begin{itemize}
\item 正規化はmigrationにも効く
\end{itemize}
\end{frame}
\begin{frame}[label={sec:orgb54e629}]{データベース設計と正規化}
\begin{itemize}
\item なぜ正規化するか
\item 正規化の目的
\end{itemize}
\end{frame}
\begin{frame}[label={sec:org4a61853}]{正規化の方法}
\begin{itemize}
\item 第一正規化
\item 第二正規化
\item 第三正規化
\end{itemize}
\end{frame}
\begin{frame}[label={sec:org3cecd87}]{正規化の効果}
\begin{itemize}
\item データ空間効率
\item 依存関係とスキーママイグレーション
\end{itemize}
\end{frame}
\section{django apps}
\label{sec:org593f997}
\begin{frame}[label={sec:orge28a8d4}]{モデルと機能}
\begin{itemize}
\item モデルの置き場所
\item 機能(views)の置き場所
\end{itemize}
\end{frame}
\begin{frame}[label={sec:org75ea683}]{データのライフサイクル}
\begin{itemize}
\item INSERT,UPDATE,DELETE
\end{itemize}
\end{frame}
\end{document}